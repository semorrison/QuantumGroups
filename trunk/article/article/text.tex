\section{todo}
\begin{itemize}
\item  See \url{http://arxiv.org/abs/0803.2778} for representations of $B_3$
coming from $\uqsl{2}$.
\item Cite \cite{MR1026957}?
\item Read \cite{MR1099254}, cite them, and work out how to use what they
do!
\end{itemize}

\section{Introduction}
What are the Reshetikhin-Turaev invariants \cite{reshetikhin, MR1025161, MR1036112, MR939474} of links coming
from quantum groups? For each quantum group $\uqg$ (by which we mean the
quantised universal enveloping algebra of a complex simple Lie algebra
$\mathfrak{g}$, see below), we have a function

$$(\text{framed links, with components labelled by irreps of $\uqg$}) \To
\Integer[q,q^{-1}].$$

In this paper, I describe how one computes these invariants. In
particular, I'll tell you just enough mathematics for the definition, but
much more importantly, I'll tell you how to \emph{actually} compute them,
by showing you how to use a \MMA package I've written called
\package{QuantumGroups}.

In fact, the package does much more than just compute quantum knot
invariants. Subject to quite restrictive practical
limitations,\footnote{My code is inefficient, the algorithms are slow,
and the computations are difficult!} the \pkg can
\begin{itemize}
\item Calculate dimensions of weight spaces and invariant spaces of
tensor products of arbitrary highest weight representations, using a
combinatorial model.
\item Produce matrices representing the action of the generators of the quantum group
$\uqg$ on an arbitrary highest weight representation.
\item Calculate bases for the invariants spaces inside tensor
products of representations, or bases for intertwining maps between two
such tensor products.
\item Calculate the action of the universal $R$-matrix on pairs of
representations.
\end{itemize}

By the end, you'll understand how to answer questions like:
\begin{quote}
What is the invariant of the knot $8_{19}$ \todo{picture!}, labelled by
the 14 dimensional irrep of $G_2$?
\end{quote}
(For the really impatient, one way is to download the \package{KnotTheory}
\MMA package from \url{http://katlas.org/}, and enter\footnote{Don't type `{\color{blue}In[1]$:=$}'; \MMA will add this itself. See \S ?? for more details.} the following in \MMA:
\begin{mma}
\begin{inm}<<KnotTheory`\end{inm}
\begin{printm}Loading KnotTheory` version of January 18, 2008, 18:17:28.7446. \\
Read more at \url{http://katlas.org/wiki/KnotTheory}.
\end{printm}
\begin{inm}V = Irrep[G2][\{0,1\}]; K = Knot[8, 19];
QuantumKnotInvariant[G2, V][K][q]\end{inm} \todo{check this works!}
\begin{outm}???\end{outm}
\end{mma}

\section{What's already done?}
The Reshetikhin-Turaev invariants have been around for quite a while, but
there hasn't been a significant tabulation of calculations, or a general
purpose program to compute them. In this section I'll summarise what's
already known. I'll concentrate on mentioning general purpose programs,
which work for arbitrary links (or perhaps just knots). There's certainly
more to say for many particular families of links.

The Jones polynomial \cite{???} is the first interesting special case,
when $\mathfrak{g}=\sl{2}$, and each component of the link is labelled
with the two dimensional representation. Of course programs to compute
this abound \cite{???}, as do tabulations of the invariants \cite{???}.
From the Jones polynomial, we can generalise in two directions:
\begin{enumerate}
\item Labelling the link with other irreps of $\uqsl{2}$. When all the
labels are the $n+1$ dimensional irrep, this is called the $n$-th
coloured Jones polynomial of the link.
\item Using the quantum group $\uqsl{n}$, and labelling each component by
the standard $n$ dimensional irrep.
\end{enumerate}

Again, there are many programs available which calculate both of these
invariants, and many tabulations. It's a little unusual to see direct
discussion of the invariant coming from the standard representation of
$\uqsl{n}$, however, because it turns out that these invariants, for
varying $n$, all fit together as a two variable polynomial, the HOMFLYPT
polynomial \cite{MR776477}. In particular,

$$\text{HOMFLYPT}_K(q^n, q-q^{-1}) = \frac{RT_{\uqsl{n}, \Complex^n}(K)(q)}{\qiq{n}{q}}.$$
\todo{explain how to check this, and where it's proved! Perhaps \cite{MR1024455}? But I can't get it.}

Thus to find programs or tables of these invariants, you're for the most
part better off looking for the HOMFLYPT invariant. One notable exception
is a program available in the \package{KnotTheory} \MMA package \cite{???},
which makes a direct calculation of the $\uqsl{3}$ invariant, via
Kuperberg's spider \cite{???}.

Next, the two variable Kauffman polynomial \cite{MR958895} simultaneously
captures all the Reshetikhin-Turaev invariants for the standard
representations of the quantum groups $U_q(\mathfrak{so}(n))$, $n \geq
5$, and $U_q(\mathfrak{sp}(n))$, $n \geq 4$, see \cite{MR939474, MR1090432}.
We'll write $V^\natural$ for the standard representation, and (see below)
$B_n$ to mean $\mathfrak{so}(2n+1)$ acting on $\Complex^{2n+1}$, $C_n$ to
mean $\mathfrak{sp}(n)$ acting on $\Complex^{2n}$, and $D_n$ to mean
$\mathfrak{so}(2n)$ acting on $\Complex^{2n}$. The quantum integers are
denoted $\qiq{k}{q} = \frac{q^k-q^{-k}}{q-q^{-1}}$.

\todo{Two thing might be wrong here: $(-1)^{\card{L}}$ might actually be $(-1)^{\card{L}+1}$, and the right hand sides are all framing independent!}
\begin{align}
\label{eq:kauffman-B}%
\text{Kauffman}_L(-i q^{4n}, i (q^2-q^{-2})) & = (-1)^{\card{L}-1} \frac{RT_{B_n, V^\natural}(L)(q)}{\qiq{2n}{q^2} + 1} \\
\label{eq:kauffman-C}%
\text{Kauffman}_L(i q^{2n+1}, i (q-q^{-1})) & =  -\frac{RT_{C_n, V^\natural}(L)(q)}{\qiq{2n+1}{q} - 1} \\
\label{eq:kauffman-D}%
\text{Kauffman}_L(-i q^{2n-1}, i (q-q^{-1})) & = (-1)^{\card{L}-1} \frac{RT_{D_n, V^\natural}(L)(q)}{\qiq{2n-1}{q} + 1}
\end{align}

Later, in \S \ref{??}, we'll show how to check these identities for particular knots and particular values of $n$, using the \pkg. \todo{}

Noticing that $\text{Kauffman}_L(a,z) = \text{Kauffman}_L(-a,-z)$, we
observe that one can pass between the expressions in Equations
\ref{eq:kauffman-C} and \ref{eq:kauffman-D} by replacing $n$ with $-n$
and $q$ with $q^{-1}$. This amusing fact is a presumably a reflection of
an identity in Vassiliev theory, due to Kontsevich, described in
\cite[Exercise 6.37]{MR1318886}. The Reshetikhin-Turaev invariant in types $B$ and $D$ can also be written more naturally (i.e. without signs)
as a specialisation of the Dubrovnik normalisation \cite{MR966143,MR980759} of the Kauffman polynomial.

\todo{mention cabling formulas. what happens outside of type A??}
\todo{examples in MMA}

\section{Installing the \pkg}

\section{Combinatorial representation theory}
A simple Lie group is a thing of beauty. In order to pass from a simple Lie group to the corresponding quantum group, however, we will need to commit
an act of violence. An iridescent butterfly will be netted, pinned down, and hidden in a glass case. Working only from some combinatorial data
describing that sad shadow of the original, we'll define the quantum group.

For the details of the next few paragraphs, we refer to any of several excellent texts \cite{?,?,?}, and assume that at least the outline is familiar.

Given a simple Lie group $G$, we begin by picking a maximal torus $T \subset G$, and then ...

If alternatively we start from the Lie algebra $\mathfrak{g}$ of $G$, the maximal torus $T$ corresponds to a maximal abelian sub-algebra, called a Cartan
subalgebra $mathfrak{h}$. ...

This Cartan matrix, or its corresponding encoding in a Dynkin diagram, is enough to capture the original Lie group up to isomorphism. The possible
Cartan matrices can be classified, for example in \cite{humphreys}, and this famous classification is summarised below.

A complex simple Lie algebra
$\mathfrak{g}$ is isomorphic to one of the following prototypes.
\begin{itemize}
\item $\mathfrak{sl}_{n+1}$, $n \geq 1$, also called $A_n$, with Dynkin diagram
$$\mathfig{0.35}{dynkin/An},$$ and Cartan matrix $$\left(
\begin{array}{rrrrrrr}
 2 & -1 & 0 & 0 & 0 &\cdots & 0 \\
 -1 & 2 & -1 & 0 & 0 &\cdots & 0 \\
 0 & -1 & 2 & -1 & 0 & \cdots & 0 \\
 \cdots & \cdots & \cdots & \cdots & \cdots & \cdots & \cdots \\
 0 & 0& \cdots & -1 & 2 & -1 & 0 \\
 0 & 0 &\cdots & 0 & -1 & 2 & -1 \\
 0 & 0 &\cdots & 0 & 0 & -1 & 2
\end{array}
\right),$$
\item $\mathfrak{so}_{2n+1}$, $n \geq 2$, also called $B_n$, with Dynkin diagram
$$\mathfig{0.35}{dynkin/Bn},$$ and Cartan matrix
$$\left(
\begin{array}{rrrrrrr}
 2 & -1 & 0 & 0 & 0 & \cdots & 0 \\
 -1 & 2 & -1 & 0 & 0 & \cdots & 0 \\
 0 & -1 & 2 & -1 & 0 & \cdots &0 \\
 \cdots & \cdots & \cdots & \cdots & \cdots & \cdots & \cdots \\
 0 & 0 &\cdots & -1 & 2 & -1 & 0 \\
 0 & 0 &\cdots & 0 & -1 & 2 & -1 \\
 0 & 0 &\cdots & 0 & 0 & -2 & 2
\end{array}
\right),$$
\item $\mathfrak{sp}_{2n}$, $n \geq 3$, also called $C_n$, with Dynkin diagram
$$\mathfig{0.35}{dynkin/Cn},$$ and Cartan matrix $$\left(
\begin{array}{rrrrrrr}
 2 & -1 & 0 & 0 & 0 & \cdots & 0 \\
 -1 & 2 & -1 & 0 & 0 & \cdots & 0 \\
 0 & -1 & 2 & -1 & 0 & \cdots &0 \\
 \cdots & \cdots & \cdots & \cdots & \cdots & \cdots & \cdots \\
 0 & 0 &\cdots & -1 & 2 & -1 & 0 \\
 0 & 0 &\cdots & 0 & -1 & 2 & -2 \\
 0 & 0 &\cdots & 0 & 0 & -1 & 2
\end{array}
\right),$$
\item $\mathfrak{so}_{2n}$, $n \geq 4$, also called $D_n$, with Dynkin
diagram $$\mathfig{0.40}{dynkin/Dn},$$ and Cartan matrix
$$\left(
\begin{array}{rrrrrrr}
 2 & -1 & 0 & 0 & 0 & \cdots &  0 \\
 -1 & 2 & -1 & 0 & 0 & \cdots & 0 \\
 0 & -1 & 2 & -1 & 0 & \cdots & 0 \\
 \cdots & \cdots & \cdots & \cdots & \cdots & \cdots & \cdots \\
 0 & 0 &\cdots &  2 & -1 & 0 & 0  \\
 0 & 0 &\cdots & -1 & 2 & -1 & -1 \\
 0 & 0 &\cdots &0 & -1 & 2 & 0 \\
 0 & 0 &\cdots &0 & -1 & 0 & 2
\end{array}
\right),$$ along with the 5 sporadic examples,
\item $E_6$, $E_7$ and $E_8$, with Dynkin diagrams and Cartan matrices
\begin{align*}
E_6 & = \mathfig{0.30}{dynkin/E6}, \\
   & = \left(
\begin{array}{rrrrrr}
 2 & 0 & -1 & 0 & 0 & 0 \\
 0 & 2 & 0 & -1 & 0 & 0 \\
 -1 & 0 & 2 & -1 & 0 & 0 \\
 0 & -1 & -1 & 2 & -1 & 0 \\
 0 & 0 & 0 & -1 & 2 & -1 \\
 0 & 0 & 0 & 0 & -1 & 2
\end{array}
\right) \\
E_7 & = \mathfig{0.35}{dynkin/E7}, \\
   & \left(
\begin{array}{rrrrrrr}
 2 & 0 & -1 & 0 & 0 & 0 & 0 \\
 0 & 2 & 0 & -1 & 0 & 0 & 0 \\
 -1 & 0 & 2 & -1 & 0 & 0 & 0 \\
 0 & -1 & -1 & 2 & -1 & 0 & 0 \\
 0 & 0 & 0 & -1 & 2 & -1 & 0 \\
 0 & 0 & 0 & 0 & -1 & 2 & -1 \\
 0 & 0 & 0 & 0 & 0 & -1 & 2
\end{array}
\right) \\
E_8 & = \mathfig{0.40}{dynkin/E8}, \\
 & = \left(
\begin{array}{rrrrrrrr}
 2 & 0 & -1 & 0 & 0 & 0 & 0 & 0 \\
 0 & 2 & 0 & -1 & 0 & 0 & 0 & 0 \\
 -1 & 0 & 2 & -1 & 0 & 0 & 0 & 0 \\
 0 & -1 & -1 & 2 & -1 & 0 & 0 & 0 \\
 0 & 0 & 0 & -1 & 2 & -1 & 0 & 0 \\
 0 & 0 & 0 & 0 & -1 & 2 & -1 & 0 \\
 0 & 0 & 0 & 0 & 0 & -1 & 2 & -1 \\
 0 & 0 & 0 & 0 & 0 & 0 & -1 & 2
\end{array}
\right),
\end{align*}
\item $F_4$, with Dynkin diagram $$\mathfig{0.20}{dynkin/F4},$$ and Cartan matrix $\left(
\begin{array}{rrrr}
 2 & -1 & 0 & 0 \\
 -1 & 2 & -1 & 0 \\
 0 & -2 & 2 & -1 \\
 0 & 0 & -1 & 2
\end{array}
\right)$,
and finally
\item $G_2$, with Dynkin diagram $$\mathfig{0.10}{dynkin/G2},$$ and Cartan matrix $\left(
\begin{array}{rr}
 2 & -3 \\
 -1 & 2
\end{array}
\right)$.
\end{itemize}
In the \pkg, you can write these in either of two
forms, for example $A2$ or $A_2$.

... and thus every representation of $\uqg$ splits up into the
simultaneous eigenspaces of the $K_i$. These spaces are called the
`weight spaces'. A representation $V$ is a `high weight' representation
if there is a weight vector $v$ so that $V = \uqg^-(v)$.

The finite dimensional irreps of $\uqg$ are all high weight
representations, and for each dominant weight there is a single
isomorphism class of such irreps. We'll thus write $V_\lambda$ to denote
`the' representation with high weight $\lambda$.

The two standard problems in combinatorial representation theory are
determining the weight multiplicities of an irrep (that is, determining
the dimensions of the weight spaces), and determining the multiplicities
of irreps inside the tensor product of two given irreps.

Both of these problems can be answered by using `Littelmann paths',
\cite{???}, and the \pkg exposes these algorithms
as in the examples\footnote{Symbols such as $\tensor, \directSum$ and
$\Complex$ can be entered in Mathematica by typing \code{<esc>c*<esc>},
\code{<esc>c+<esc>} and \code{<esc>dsC<esc>} respectively.} below:
\begin{mma}
\begin{inm}
DecomposeRepresentation[A2][Irrep[A2][\{1,0\}] $\tensor$
Irrep[A2][\{0,1\}]]
\end{inm}
\begin{outm}
$\Complex$ $\directSum$ Irrep[A2][\{1,1\}]
\end{outm}
\begin{inm}
WeightMultiplicities[F4,Irrep[F4][\{0,0,0,1\}]]
\end{inm}
\begin{outm}
???
\end{outm}
\end{mma}

\section{Explicit representations}
Perhaps the most important function in the \pkg is
\code{MatrixPresentation}, which produces explicit matrices representing
the action of the quantum group generators on a representation.

In order to understand how these are produced, we need to make use of the
following two results:

\begin{fact}
\label{fact:fundamental}
Every irrep of $\uqg$ is a subrepresentation of some tensor product
of fundamental representations.
\end{fact}

\begin{fact}
\label{fact:generators}
Every fundamental representation is subrepresentation of some
tensor product of `minuscule representations' and `short root
representations'.
\end{fact}

The first fact is trivial; to produce the irrep with highest weight
$\lambda = (\lambda_1, \lambda_2, \ldots, \lambda_n)$, pick high weight
vectors $v_i$ in each fundamental representation $V_{e_i}$, and look at
$\uqg^-\left(\Tensor_{i=1}^n  v_i^{\tensor \lambda_i}\right) \subset
\Tensor_{i=1}^n  V_{e_i}^{\tensor \lambda_i}$. This is an irrep,
generated by a high weight vector, and so must be what we want.

We'll explain now what `minuscule' and `short root' representations are,
and explain an easy proof of the second fact. It also appears as Proposition 5A.10 of \cite{jantzen}, but we'll give
a `case-by-case' argument, building on the first fact about fundamental representations.

Although it is unsurprising, Fact \ref{fact:generators} is essential to what follows. The minuscule and short root representations can be
presented completely explicitly, and we'll use these to build up
everything else.

There are several equivalent characterisations of a minuscule
representation. The simplest to state is ... ??? (depends on whether
we've mentioned the weyl group!) \cite[\S 5A.1]{jantzen} \cite[ch. VI, \S 1, exercise 24]{bourbaki}

A `short root representation' is an irrep whose highest weight is a root of the Lie algebra, in the dominant Weyl chamber, of minimal length. In the
simply laced cases (types $A$, $D$ and $E$), all roots are the same length, so any dominant root gives a short root representation. \cite[\S 5A.2]{jantzen}

Minuscule representations must be fundamental representations, but the converse is not true. The
following representations are minuscule:
\begin{center}
\begin{tabular}{c|l|l}
 $\Gamma$ & minuscule representations & $V_\lambda$ \\
 \hline%
 $A_n$ & all fundamental representations & $\lambda = e_i$, $1 \leq i \leq n$ \\
 $B_n$ & the $2^n$-d representation      & $\lambda = e_n$ \\
 $C_n$ & the $2n$-d defining representation             & $\lambda = e_1$ \\
 $D_n$ & the $2n$-d defining and `spin' representations & $\lambda = e_1, e_{n-1}, e_{n}$ \\
 $E_6$ & both $27$-d representations                    & $\lambda = e_1$ or $e_6$ \\
 $E_7$ & the $56$-d representation                      & $\lambda = e_7$ \\
 $E_8$ & none                             & \\
 $F_4$ & none & \\
 $G_2$ & none &
\end{tabular}
\end{center}

Every fundamental representation of $F_4$ has weight multiplicities, so cannot be minuscule.
In $G_2$, $\{1,0\}$ has no multiplicities, but a nonzero $\{0,0\}$ weight space, which is outside the orbit of the high weight.
The fundamental representations of $E_6$ with highest weight $\lambda = e_{2 \leq k \leq 5}$ have multiplicities, as do the
fundamental representations of $E_7$ with highest weight $\lambda = e_{1 \leq k \leq 6}$ and all the fundamental representations of $E_8$.
(Note that it's note necessary to calculate the entire weight multiplicities of a representation to determine that at least one weight has multiplicity!
Otherwise is would be very difficult to determine the minuscule representations of $E_8$!)

From minuscule representations, we can build up others. In the type $A$ case, nothing further is needed; every fundamental
representation is minuscule. In the type $B$ case, we see that
$$V_{e_n}^{\tensor 2} \Iso \Complex \directSum (\DirectSum_{k=1}^{n-1} V_{e_k}) \directSum V_{2e_n},$$
and so every fundamental representation is contained in some tensor power (in particular the tensor square) of minuscule representations.
\todo{but that's not actually what we do for $B_n$!}
In the type $C$ case, we find that for $2 \leq k \leq n$, the tensor power $V_{e_1}^{\tensor k}$ contains one copy of the fundamental representation
$V_{e_k}$.
In the type $D$ case, it's a little complicated! ???

Finally, of the exceptional groups, we clearly can't get anywhere at all using minuscule representations for $E_8$, $F_4$ or $G_2$.

In $E_6$,
we find that $V_{e_1}^{\tensor 2}$ contains a copy of $V_{e_3}$, $V_{e_1} \tensor V_{e_6}$ contains a copy of $V_{e_2}$, and $V_{e_6}^{\tensor 2}$ contains a copy of $V_{e_5}$.
That gets us almost all the way there --- happily, then, we find a copy of $V_{e_4}$ inside $V_{e_1} \tensor V_{e_3}$, and hence inside $V_{e_1}^{\tensor 3}$.
These observations show that every representation of $E_6$ can be found inside tensor products of the minuscule representations.

In $E_7$, we have
\begin{align*}
V_{e_7}^{\tensor 2} & \Iso \Complex \directSum V_{e_1} \directSum V_{e_6} \directSum V_{2e_7} \\
                    & \supset V_{e_1}, V_{e_6} \\
V_{e_1}^{\tensor 2} & \Iso \Complex \directSum V_{e_1} \directSum V_{e_3} \directSum V_{e_6} \directSum V_{2e_1} \\
                    & \supset V_{e_3} \\
V_{e_1} \tensor V_{e_7} & \Iso V_{e_2} \directSum V_{e_7} \directSum V_{e_1+e_7} \\
                        & \supset V_{e_2} \\
V_{e_1} \tensor V_{e_2} & \Iso V_{e_2} \directSum V_{e_5} \directSum V_{e_7} \directSum V_{e_1 + e_2} \directSum V_{e_1 + e_7} \\
                        & \supset V_{e_5} \\
V_{e_1} \tensor V_{e_3} & \Iso V_{e_1} \directSum V_{e_3} \directSum V_{e_4} \directSum V_{e_6} \directSum V_{e_2 + e_7}\directSum V_{e_1+e_3} \directSum V_{e_1+e_6}\directSum V_{2e_1}\\
                        & \supset V_{e_4},
\end{align*}
and again, we find everything inside tensor products of minuscules, and in particular in a tensor power of the unique minuscule representation!

The representations $V_{e_8}$ of $E_8$, $V_{e_4}$ of $F_4$ and $V_{e_1}$ of $G_2$ are short root representations. We can find all fundamental
representations inside tensor powers of these, as follows.

In $E_8$, we have
\begin{align*}
V_{e_8}^{\tensor 2}     & \Iso \Complex \directSum V_{e_1} \directSum V_{e_7} \directSum V_{e_8} \directSum V_{2e_8} \\
                        & \supset V_{e_1} \directSum V_{e_7} \\
V_{e_1} \tensor V_{e_8} & \Iso V_{e_1} \directSum V_{e_2} \directSum V_{e_7} \directSum V_{e_8} \directSum V_{e_1+e_8} \\
                        & \supset V_{e_2} \\
V_{e_1}^{\tensor 2}     & \Iso \Complex \directSum V_{e_1} \directSum V_{e_2} \directSum V_{e_3} \directSum V_{e_6} \directSum V_{e_7} \directSum V_{e_8} \directSum V_{2e_1} \directSum V_{e_1+e_8} \directSum V_{2e_8} \\
                        & \supset V_{e_3} \directSum V_{e_6} \\
V_{e_2} \tensor V_{e_1} & \Iso V_{e_1} \directSum V_{e_2} \directSum V_{e_3} \directSum V_{e_5} \directSum V_{e_6} \directSum V_{e_7} \directSum V_{e_8} \directSum \\
                        & \qquad  \directSum V_{2e_1} \directSum V_{2e_8} \directSum V_{e_1+e_2} \directSum V_{e_1+e_7} \directSum 2V_{e_1+e_8} \directSum V_{e_2+e_8} \directSum V_{e_7+e_8}\\
                        & \supset V_{e_5} \\
V_{e_3} \tensor V_{e_1} & \Iso V_{e_1} \directSum V_{e_2} \directSum 2V_{e_3} \directSum V_{e_4} \directSum V_{e_5} \directSum V_{e_6} \directSum V_{e_7} \directSum V_{2e_1} \directSum \\
                        & \qquad \directSum 2V_{e_1+e_2} \directSum V_{e_1+e_3} \directSum V_{e_1+e_6} \directSum 2V_{e_1+e_7} \directSum 2V_{e_1+e_8} \directSum V_{e_2+e_7} \directSum \\
                        & \qquad \directSum 2V_{e_2+e_8} \directSum V_{e_3+e_8} \directSum V_{e_6+e_8} \directSum V_{e_7+e_8} \directSum V_{2e_1+e_8} \directSum V_{e_1+2e_8}\\
                        & \supset V_{e_4}
\end{align*}
(Note that in calculating the above, although decomposing $V \tensor W$ gives the same direct summands as decomposing $W \tensor V$, the two calculations
might take dramatically different amounts of time, because of how the Littelmann path algorithm works. In particular, we need to know the entire weight multiplicities of the second factor, but nothing except the highest weight of the first.
This fact explains why we've written the tensor products above one way and not the other.)

In $F_4$,
\begin{align*}
V_{e_4}^{\tensor 2}     & \\
V_{e_1}^{\tensor 2}     & \\
\intertext{and finally in $G_2$}
V_{e_1}^{\tensor 2} & = \Complex \directSum V_{e_2} \directSum V_{2e_1} ???
\end{align*}


Here is what minuscule representations look like

Here is what short root representations look like, cf Jantzen.

It is invoked as \code{MatrixPresentation$[\Gamma][Z][V, \lambda,
\beta]$}. Here
\begin{defn}
\item[$\Gamma$] is the Cartan type, see \S \ref{sec:cartan-types}.
\item[$Z$] is a generator of the quantum group $\Gamma$, that is
$X_i^\pm$ or $K_i$, for $1 \leq i \leq \rank{\Gamma}$. Compositions of
generators, in the notation of \ref{sec:composition}, and linear
combinations, are also allowed. (Linear combinations must be homogeneous
with respect to the weight grading.)
\item[$V$] is a representation, in the notation of \S
\ref{sec:representations}.
\item[$\lambda$] is a weight, in the notation of \S \ref{sec:weights};
that is, a vector of integers, giving the weight as a linear combination
of fundamental weights.
\item[$\beta$] is a symbol specifying the basis to use. Possible options are described in \S \ref{sec:bases}, but nearly always you'll use
\code{FundamentalBasis}.
\end{defn}

\subsection{Bases}
The function \code{MatrixPresentation} takes an argument specifying the
desired basis. In the current implementation, there is only one useful
value -- the symbol \code{FundamentalBasis}. While we give a description
of how this basis is recursively defined below, essentially it depends on
many minor details of the implementation. One should not depend on any
particular properties of this basis!

A hypothetical future versions of the \pkg might allow the use of
the symbols \code{GelfandTsetlinBasis} and \code{CanonicalBasis}, with
the obvious results.\footnote{Gelfand-Tsetlin bases are only projectively
canonical.} Code implementing Gelfand-Testlin bases for type $A$ exists, but is not
currently part of the package. Anyone interested in adding support for
canonical bases should certainly contact me!

\subsection{Examples}
We conclude this section with several examples.

\begin{mma}
\begin{inm}V = Irrep[A1][\{5\}];
Table[MatrixPresentation[G2][$X_1^+$][V, $\{\lambda\}$, FundamentalBasis], $\{\lambda, -5, 5, 2\}$]\end{inm}
\begin{outm}???\end{outm}
\begin{inm}V = Irrep[G2][\{1, 0\}];
MatrixPresentation[G2][$X_1^+$][V, \{0, 0\}, FundamentalBasis]\end{inm}
\begin{outm}???\end{outm}
\begin{inm}MatrixPresentation[G2][$X_2^+$][V, \{0, 0\}, FundamentalBasis]\end{inm}
\begin{outm}???\end{outm}
\begin{inm}V = Irrep[A3][\{2, 2, 2\}];
MatrixPresentation[A3][$X_1^+$][V, \{0, 0, 0\}, FundamentalBasis]\end{inm}
\begin{outm}???\end{outm}
\end{mma}


\section{Invariant vectors and Hom-spaces}


\section{$R$-matrices and quantum knot invariants}
\subsection{Action of the Coxeter braid group on the quantum group}
Next, we need to make use of the `Coxeter braid group' associated to our quantum group, and the action of this braid group on the quantum group itself.
This is the quantum analogue of the Weyl group action on the classical universal enveloping algebra. It's important to remember,
as we approach defining quantum knot invariants, that although quantum groups outside of type $A$ have Coxeter braid groups which are not
the usual Artin braid groups, it is still the usual Artin braid group which acts on tensor products of representations!

The braid group $\mathcal{B}_\mathfrak{g}$ associated to a complex simple Lie algebra $\mathfrak{g}$ of rank $n$ is

\begin{equation*}
\mathcal{B}_\mathfrak{g} =
\relations{
    T_i, 1 \leq i \leq n
}{
    \begin{array}{c}
        T_i T_j T_i T_j \cdots = T_j T_i T_j T_i \cdots \\
        \text{with $2, 3, 4$ or $6$ factors on each side,} \\
        \text{when $a_{ij} a_{ji}$ = $0, 1, 2$ or $3$ respectively.}
    \end{array}
}
\end{equation*}

It is always infinite, and collapses to the Weyl group $W_\mathfrak{g}$ for $\mathfrak{g}$ if we impose the additional relations $T_i^2 = 1$.

The following formulas are simply translated from \cite[\S8.1A]{CP}\footnote{In \cite{CP}, the authors write the action on a slightly different presentation of the quantum group,
over $\Complex[[h]]$, and I'm using the corresponding action on the quantum group defined over $\Complex(q)$}:

\todo{reduced powers!, $d_i$}

\begin{align*}
T_i(X_i^+) & = - X_i^- K_i \\
T_i(X_i^-) & = - K_i X_i^+ \\
T_i(K_j)   & =  K_j K_i^{-a_{ij}} \\
T_i(X_j^+)   & = \sum_{r=0}^{-a_{ij}} (-1)^{r-a_{ij}} q^{-r d_i} (X_i^+)^{(-a_{ij}-r)} X_j^+ (X_i^+)^{(r)} \qquad \text{if $i \neq j$} \\
T_i(X_j^-)   & = \sum_{r=0}^{-a_{ij}} (-1)^{r-a_{ij}} q^{+r d_i} (X_i^-)^{(r)} X_j^- (X_i^-)^{(-a_{ij}-r)} \qquad \text{if $i \neq j$}
\end{align*}

As the authors of \cite{CP} point out, the claim that this defines an action of $\mathcal{B}_\mathfrak{g}$ by algebra automorphisms can be checked explicitly. Happily,
at least for a given $\mathfrak{g}$, the \pkg really can do this check. \todo{see, something}.

\subsection{Quantum positive roots}
The `quantum positive roots', which are elements of $\uqg$, are now defined as the action of certain words in the Coxeter braid group on certain of the
generators $X_i^+$. It's possible to write the longest word in the Weyl group as a minimal length product of simple reflections in several ways, and
we'll use `the long word decomposition' to mean the lexicographically least one:

$$ w_0 = s_{i_1} s_{i_2} \cdots s_{i_r}.$$

We then define, following \cite{CP}.

$$X_{\uqg, 1}^\pm = X_{i_1}^\pm, X_{\uqg, 2}^\pm = T_{i_1}(X_{i_2}^\pm), \ldots, X_{\uqg, r}^\pm = T_{i_1}T_{i_2}\cdots T_{i_{r-1}}(X_{i_r}^\pm).$$

Using a different decomposition of the longest word gives a different set of quantum positive roots, unlike in the classical case! \cite[\S8.1B]{CP}

We note that the lexicographically least longest word decompositions can be summarised as follows:
\begin{center}
\begin{tabular}{c|l}
$A_1$ & $1$ \\
$A_2$ & $1,2,1$ \\
$A_n$ & $w_0(A_{n-1}),n,n-1,\ldots,1$ \\
$B_2$ & $1,2,1,2$ \\
$B_3$ & $1,2,1,3,2,1,3,2,3$ \\
$B_4$ & $1,2,1,3,2,1,4,3,2,1,4,3,2,4,3,4$ \\
$B_n$ & $w_0(A_n),(n+1)-\operatorname{rev}(w_0(A_{n-1}))$, or equivalently\\
      & $w_0(A_{n-1}),(n+1)-\operatorname{rev}(w_0(A_n))$ \\
$C_n$ & same as $B_n$ \\
$D_4$ & $1,2,1,3,2,1,4,2,1,3,2,4$ \\
$D_5$ & $1,2,1,3,2,1,4,3,2,1,5,3,2,1,4,3,2,5,3,4$ \\
$D_n$ & $w_0(A_{n-1}),(n,(n-2,n-3,\ldots,1)),(n-1,(n-2,n-3,\ldots,2)),$ \\
      & $\qquad(n,(n-2,n-3,\ldots,3)),\ldots,(n\text{ or }n-1,(n-2)),(n-1\text{ or }n)$ \\
$E_6$ & $1,2,3,1,4,2,3,1,4,3,5,4,2,3,1,4,3,5,4,2,6,5,4,2,3,1,4,3,5,4,2,6,5,4,3,1$\\
$E_7$ & $w_0(E_6), 7, 6, 5, 4, 2, 3, 1, 4, 3, 5, 4, 2, 6, 5, 4, 3, 1, 7, 6, 5, 4, 2, 3, 4, 5, 6, 7$ \\
$E_8$ & $w_0(E_7), 8, 7, 6, 5, 4, 2, 3, 1, 4, 3, 5, 4, 2, 6, 5, 4, 3, 1, 7, 6, 5, 4, 2, 3, 4, 5, 6, 7, $ \\
      & $\qquad    8, 7, 6, 5, 4, 2, 3, 1, 4, 3, 5, 4, 2, 6, 5, 4, 3, 1, 7, 6, 5, 4, 2, 3, 4, 5, 6, 7, 8$ \\
$F_4$ & $1,2,1,3,2,1,3,2,3,4,3,2,1,3,2,3,4,3,2,1,3,2,3,4$ \\
$G_2$ & $1,2,1,2,1,2$
\end{tabular}
\end{center}
The expression for $D_n$, which is quite complicated, includes some extra parentheses to help you see the pattern. Whether $w_0(D_n)$ ends with an
$n-1$ or an $n$ depends on whether $n$ is odd or even, respectively.

The \package{QuantumGroups} package can tell you these decompositions, as follows
\begin{mma}
\begin{inm}LongestWordDecomposition[D4]\end{inm}
\begin{outm}\{1, 2, 1, 3, 2, 1, 4, 2, 1, 3, 2, 4\}
\end{outm}
\end{mma}
although it's worth admitting that it's not actually calculating these from scratch. In principle it can, and will produce a list of reflection matrices
representing the Weyl group elements with respect to the fundamental basis, as for example
\begin{mma}
\begin{inm}WeylGroup[A2]\end{inm}
\begin{outm}$\left\{\left(\begin{array}{rr} 1 & 0 \\ 0 & 1\end{array}\right),\left(\begin{array}{rr} -1 & 0 \\ 1 & 1\end{array}\right),\left(\begin{array}{rr} 1 & 1 \\ 0 & -1\end{array}\right),\right.$
$\qquad\qquad\left.\left(\begin{array}{rr} -1 & -1 \\ 1 & 0\end{array}\right),\left(\begin{array}{rr} 0 & 1 \\ -1 & -1\end{array}\right),\left(\begin{array}{rr} 0 & -1 \\ -1 & 0\end{array}\right)\right\}$
\end{outm}
\end{mma}
but in practice it's inefficient enough that, for example, finding the longest word decomposition for $E_8$ is completely impractical. I've included
decompositions calculated by the ``chevie'' package in ``GAP''.

By default, the \package{QuantumGroups} package leaves quantum roots unevaluated, and you need to explicitly apply the function \code{ExpandQuantumRoots}
in order to reexpress them in terms of the algebra elements $X_i^\pm$. Thus, for example, we have
\begin{mma}
\begin{inm}ExpandQuantumRoots[A2] /@ $\{ X_{A2,1}^+, X_{A2,2}^+, X_{A2,3}^+ \}$
\end{inm}
\begin{outm}$\left\{\left(X_1\right){}^+,-\left(X_1\right){}^+\left(X_2\right){}^+ + q^{-1}\left(X_2\right){}^+\left(X_1\right){}^+,\left(X_2\right){}^+\right\}$
\end{outm}
\end{mma}
agreeing with \cite[Example 8.1.5]{CP}, and the most complicated of the quantum positive roots for $D_4$,
\begin{mma}
\begin{inm}ExpandQuantumRoots[D4][$X_{D4,7}^+$]
\end{inm}
\begin{outm}$-q^{-1}X_1^+ X_2^+ X_4^+ X_2^+ X_3^+ + q^{-2}X_1^+ X_2^+ X_4^+ X_3^+ X_2^+ + q^{-2}X_2^+ X_1^+ X_4^+ X_2^+ X_3^+$
$\qquad  -q^{-3}X_2^+ X_1^+ X_4^+ X_3^+ X_2^+ +X_2^+ X_3^+ X_1^+ X_2^+ X_4^+ - q^{-1}X_2^+ X_3^+ X_2^+ X_1^+ X_4^+$
$\qquad  - q^{-1}X_2^+ X_3^+ X_4^+ X_1^+ X_2^+ + q^{-2}X_2^+ X_3^+ X_4^+ X_1^+ X_1^+ -q^{-1}X_3^+ X_2^+ X_1^+ X_2^+ X_4^+$
$\qquad + q^{-2}X_3^+ X_2^+ X_2^+ X_1^+ X_4^+ + q^{-2}X_3^+ X_2^+ X_4^+ X_1^+ X_2^+ -q^{-3}X_3^+ X_2^+ X_4^+ X_2^+ X_1^+$
$\qquad +q^{-2}X_4^+ X_1^+ X_2^+ X_2^+ X_3^+ - q^{-3}X_4^+ X_1^+ X_2^+ X_3^+ X_2^+ - q^{-3}X_4^+ X_2^+ X_1^+ X_2^+ X_3^+$
$\qquad \qquad + q^{-4}X_4^+ X_2^+ X_1^+ X_3^+ X_2^+$
\end{outm}
\end{mma}

These calculations become difficult and slow, and so the \pkg includes precomputed quantum roots for everything up to rank $4$, except $F_4$. (The current version runs
for a several weeks without finishing the computation for the expansion of $X_{F_4, 13}$; suggestions for calculating these more efficiently would be
appreciated.)

\subsection{The universal $R$-matrix}
Having constructed the quantum roots, we can now make use of a formula for the universal $R$-matrix from \cite{CP}, which works for every $\mathfrak{g}$.
Unfortunately the formula as given, in \cite[Theorem 8.3.9]{CP}, is incorrect; in particular the order of a product has been reversed.
Some further translation is needed, as that formula for the universal $R$-matrix is for the quantum group defined of $\Complex[[h]]$ rather than $\Complex(q)$.
Fortunately this is already done in \S10.1.D of \cite{CP}, so we just summarise that, making the necessary correction.

For each $n\geq0$, define
\begin{equation}
\mathcal{R}^{(n)} = \sum_{\substack{t_1, \ldots, t_N \\ \sum ??? t_r = n}} \prod_{r=1}^N q_{\beta_r}^{\frac{1}{2}t_r(t_r+1)} \frac{(1-q_{\beta_r}^{-2})^{t_r}}{[t_r]_{q_{\beta_r}}!} (X_{\beta_r}^+)^{t_r} \tensor (X_{\beta_r}^-)^{t_r}.
\end{equation}
where $[k]_q! = [k]_q [k-1]_q \cdots [2]_q [1]_q$ is the quantum factorial, and $q_\beta = q^{d_\beta}$, $d_\beta = d_i$ if $\beta$ is the $i$-th simple root, and $d_\beta = \sum \alpha_i d_i$ when $\beta = \sum \alpha_i ???$ \todo{notation for simple roots and positive roots?}.
(The formula is \cite{CP} incorrectly requires that this product be written in the reverse order.)

It is clear that on any finite dimensional representation, eventually $\mathcal{R}^{(n)}$ acts by zero. Given a pair of finite dimension representations $V$ and $W$,
define $\mathcal{R}_{V,W}$ on the product of weight spaces $V_\lambda \tensor W_\mu$ by
\begin{equation}
\mathcal{R}_{V,W}(v_\lambda\tensor w_\mu) = q^{(\lambda,\mu)} \sum_{n=0}^\infty \mathcal{R}^{(n)}(v_\lambda \tensor w_\mu).
\end{equation}

Then we have \cite[Proposition 10.1.19]{CP}
\begin{prop}
The map $\mathcal{R}_{V,W}$ is invertible, and satisfies
\begin{equation*}
\mathcal{R}_{V,W} \compose \Delta(x) = \Delta^{\text{op}}(x) \compose \mathcal{R}_{V,W}
\end{equation*}
for any $x$ in the quantum group, and if we define $$\mathcal{B}_{V,W} = \sigma \compose \mathcal{R}_{V,W} : V \tensor W \To W \tensor V$$ then $\mathcal{B}_{\bullet,\bullet}$ gives a representation of the braid group
\begin{equation*}
(\Id_U \tensor \mathcal{B}_{V,W}) \compose (\mathcal{B}_{U,W}\tensor \Id_V) \compose (\Id_W\tensor\mathcal{B}_{U,V}) = (\mathcal{B}_{U,V}\tensor\Id_W) \compose (\Id_V\tensor\mathcal{B}_{U,W}) \compose (\mathcal{B}_{V,W}\tensor\Id_U)
\end{equation*}
for any finite dimensional representations $U, V$ and $W$.
\end{prop}

These maps are defined by the \pkg, and are available as functions \code{RMatrix} and \code{BraidingMap}. The two functions require slightly different syntax, and package
their answers in different ways.
The first function takes arguments
$\operatorname{RMatrix}[\Gamma, V, W, \beta, \lambda]$, and returns the matrix presentation of $\mathcal{R}_{V,W}$ acting on the $\lambda$ weight space,
using the basis described by $\beta$. The second function takes arguments $\operatorname{BraidingMap}[\Gamma, V \tensor W, \beta]$, and returns
$\mathcal{B}_{V,W}$ packaged as a \code{RepresentationTensor} expression, as described in \S \ref{??}.

As examples, we have



\begin{mma}
\begin{inm}$\text{RMatrix}[A_2, \Irrep[A_2][\{1,0\}], \Irrep[A_2][\{0,1\}], \text{FundamentalBasis}, \{0,0\}]$
\end{inm}
\begin{outm}$\left(\begin{array}{lll} \frac{1}{q^{2/3}} & 0 & 0 \\ \frac{q^2-1}{q^{5/3}} & \frac{1}{q^{2/3}} & 0 \\ \frac{1-q^2}{q^{8/3}} & \frac{q^2-1}{q^{5/3}} & \frac{1}{q^{2/3}}\end{array}\right)$\end{outm}
\begin{inm}$\text{BraidingMap}[A_2, \Irrep[A_2][\{1,0\}] \tensor \Irrep[A_2][\{0,1\}], \text{FundamentalBasis}]$
\end{inm}
\begin{outm}
\begin{equation*}
\begin{split}
\text{RepresentationTensor}[A_2,\\\text{Irrep}\left[A_2\right][\{0,1\}]\otimes
   \text{Irrep}\left[A_2\right][\{1,0\}],\text{FundamentalBasis}, \\
   \text{Irrep}\left[A_2\right][\{1,0\}]\otimes
   \text{Irrep}\left[A_2\right][\{0,1\}],\text{FundamentalBasis},\\
\left\{
\begin{array}{ll}
\{
 \{1,1\}, & \text{Matrix}\left[1,1,\left(
\begin{array}{l}
 q^{1/3}
\end{array}
\right)\right] \},\\\{
 \{2,-1\}, & \text{Matrix}\left[1,1,\left(
\begin{array}{l}
 q^{1/3}
\end{array}
\right)\right] \},\\\{
 \{-1,2\}, & \text{Matrix}\left[1,1,\left(
\begin{array}{l}
 q^{1/3}
\end{array}
\right)\right] \},\\\{
 \{0,0\}, & \text{Matrix}[3,3,\left(
\begin{array}{lll}
 \frac{1-q^2}{q^{8/3}} & \frac{q^2-1}{q^{5/3}} & \frac{1}{q^{2/3}} \\
 \frac{q^2-1}{q^{5/3}} & \frac{1}{q^{2/3}} & 0 \\
 \frac{1}{q^{2/3}} & 0 & 0
\end{array}
\right)] \},\\\{
 \{1,-2\}, & \text{Matrix}\left[1,1,\left(
\begin{array}{l}
 q^{1/3}
\end{array}
\right)\right] \},\\\{
 \{-2,1\}, & \text{Matrix}\left[1,1,\left(
\begin{array}{l}
 q^{1/3}
\end{array}
\right)\right] \},\\\{
 \{-1,-1\}, & \text{Matrix}\left[1,1,\left(
\begin{array}{l}
 q^{1/3}
\end{array}
\right)\right]\}
\end{array}
\right\}]
\end{split}
\end{equation*}
\end{outm}
\begin{inm}$\text{BraidingMap}[D_4, \Irrep[D_4][\{0,1,0,0\}] \tensor \Irrep[D_4][\{0,1,0,0\}], \text{FundamentalBasis}]$
\end{inm}
\begin{outm}
\begin{equation*}
\begin{split}
\text{RepresentationTensor}[D_4,\\\Irrep[D_4][\{0,1,0,0\}]\otimes
  \Irrep[D_4][\{0,1,0,0\}],\text{FundamentalBasis}, \\
   \Irrep[D_4][\{0,1,0,0\}]\otimes
   \Irrep[D_4][\{0,1,0,0\}],\text{FundamentalBasis},\\
\left\{
\begin{array}{ll}
\{
 \{0,-2,0,0\}, & \text{Matrix}\left[1,1,\left(
\begin{array}{l}
 q^2
\end{array}
\right)\right] \},\\\{
 \{-1,0,-1,-1\}, & \text{Matrix}\left[2,2,\left(
\begin{array}{ll}
 0 & q \\ q & q^2-1
\end{array}
\right)\right] \},\\
\cdots
\\\{
 \{0,0,0,0\}, & \text{Matrix}[40,40,\left(
\cdots
\right)] \},\\
\cdots
\end{array}
\right\}]
\end{split}
\end{equation*}\end{outm}
\end{mma}

For later:
\begin{mma}
\begin{inm}$\text{BraidingData}[A_3][\Irrep[A_3][\{0,1,0\}], 3]$
\end{inm}
\begin{outm}
??
\end{outm}
\end{mma}

\subsection{Representations of the braid group}
