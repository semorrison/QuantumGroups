\section{Introduction}
What are the Reshetikhin-Turaev invariants \cite{???} of links coming
from quantum groups? For each quantum group $\uqg$ (by which we mean the
quantised universal enveloping algebra of a complex simple Lie algebra
$\mathfrak{g}$, see below), we have a function

$$(\text{framed links, with components labelled by irreps of $\uqg$}) \To
\Integer[q,q^{-1}].$$

In this paper, I describe how one computes these invariants. In
particular, I'll tell you just enough mathematics for the definition, but
much more importantly, I'll tell you how to \emph{actually} compute them,
by showing you how to use a \MMA package called \pkg{QuantumGroups}.

In fact, the package does much more than just compute quantum knot
invariants. Subject to quite restrictive practical
limitations,\footnote{My code is inefficient, the algorithms are slow,
and the computations are difficult!} \pkg{QuantumGroups} can
\begin{itemize}
\item Calculate dimensions of weight spaces and invariant spaces of
tensor products of arbitrary highest weight representations, using a
combinatorial model.
\item Produce matrices representing the action of the generators of the quantum group
$\uqg$ on an arbitrary highest weight representation.
\item Calculate bases for the invariants spaces inside tensor
products of representations, or bases for intertwining maps between two
such tensor products.
\item Calculate the action of the universal $R$-matrix on pairs of
representations.
\end{itemize}

By the end, you'll understand how to answer questions like:
\begin{quote}
What is the invariant of the knot $8_{19}$ \todo{picture!}, labelled by
the 14 dimensional irrep of $G_2$?
\end{quote}
(For the really impatient, one way is to download the \pkg{KnotTheory}
\MMA package from \url{http://katlas.org/}, and enter\footnote{Don't type `{\color{blue}In[1]$:=$}'; \MMA will add this itself. See \S ?? for more details.} the following in \MMA:
\begin{mma}
\begin{inm}<<KnotTheory`\end{inm}
\begin{printm}Loading KnotTheory` version of January 18, 2008, 18:17:28.7446. \\
Read more at \url{http://katlas.org/wiki/KnotTheory}.
\end{printm}
\begin{inm}QuantumKnotInvariant[G2,Irrep[G2][\{0,1\}]][Knot[8,19]][q]\end{inm}
\todo{check this
works!}
\begin{outm}???\end{outm}
\end{mma}

\section{What's already done?}
The Reshetikhin-Turaev invariants have been around for quite a while, but
there hasn't been a significant tabulation of calculations, or a general
purpose program to compute them. In this section I'll summarise what's
already known. I'll concentrate on mentioning general purpose programs,
which work for arbitrary links (or perhaps just knots). There's certainly
more to say for many particular families of links.

The Jones polynomial \cite{???} is the first interesting special case,
when $\mathfrak{g}=\sl{2}$, and each component of the link is labelled
with the two dimensional representation. Of course programs to compute
this abound \cite{???}, as do tabulations of the invariants \cite{???}.
From the Jones polynomial, we can generalise in two directions:
\begin{enumerate}
\item Labelling the link with other irreps of $\uqsl{2}$. When all the
labels are the $n+1$ dimensional irrep, this is called the $n$-th
coloured Jones polynomial of the link.
\item Using the quantum group $\uqsl{n}$, and labelling each component by
the standard $n$ dimensional irrep.
\end{enumerate}

Again, there are many programs available which calculate both of these
invariants, and many tabulations. It's a little unusual to see direct
discussion of the invariant coming from the standard representation of
$\uqsl{n}$, however, because it turns out that these invariants, for
varying $n$, all fit together as a two variable polynomial, the HOMFLYPT
polynomial \cite{???}. In particular,

$$HOMFLYPT_K(q^n, q) = RT_{\uqsl{n}, \Complex^n}(K)(q).$$

Thus to find programs or tables of these invariants, you're for the most
part better off looking for the HOMFLYPT invariant. One notable exception
is a program available in the \pkg{KnotTheory} \MMA package \cite{???},
which makes a direct calculation of the $\uqsl{3}$ invariant, via
Kuperberg's spider \cite{???}.

Next, the two variable Kauffman polynomial simultaneously captures all
the Reshetikhin-Turaev invariants for the standard representations of the
quantum groups $U_q(\mathfrak{so}(n))$, $n \geq 5$, and
$U_q(\mathfrak{sp}(n))$, $n \geq 4$.

\todo{look these up, and write some formulas}

\section{Installing the \pkg{QuantumGroups} package}

\section{Combinatorial representation theory}
... Thus the possibilities for the complex simple Lie algebra
$\mathfrak{g}$ are
\begin{itemize}
\item $\mathfrak{sl}_{n+1}$, $n \geq 1$, also called $A_n$, with Dynkin diagram
???,
\item $\mathfrak{so}_{2n+1}$, $n \geq 2$, also called $B_n$, with Dynkin diagram
???,
\item $\mathfrak{sp}_{2n}$, $n \geq 3$, also called $C_n$, with Dynkin diagram
???,
\item $\mathfrak{so}_{2n}$, $n \geq 4$, also called $D_n$, with Dynkin
diagram ???, along with the 5 sporadic examples,
\item $E_6$, $E_7$ and $E_8$, with Dynkin diagrams ???,
\item $F_4$, with Dynkin diagram ???,
and finally
\item $G_2$, with Dynkin diagram ???.
\end{itemize}
In the \pkg{QuantumGroups} package, you can write these in either of two
forms, for example $A2$ or $A_2$.

... and thus every representation of $\uqg$ splits up into the
simultaneous eigenspaces of the $K_i$. These spaces are called the
`weight spaces'. A representation $V$ is a `high weight' representation
if there is a weight vector $v$ so that $V = \uqg^-(v)$.

The finite dimensional irreps of $\uqg$ are all high weight
representations, and for each dominant weight there is a single
isomorphism class of such irreps. We'll thus write $V_\lambda$ to denote
`the' representation with high weight $\lambda$.

The two standard problems in combinatorial representation theory are
determining the weight multiplicities of an irrep (that is, determining
the dimensions of the weight spaces), and determining the multiplicities
of irreps inside the tensor product of two given irreps.

Both of these problems can be answered by using `Littelmann paths',
\cite{???}, and the \pkg{QuantumGroups} package exposes these algorithms
as in the examples\footnote{Symbols such as $\tensor, \directSum$ and
$\Complex$ can be entered in Mathematica by typing \code{<esc>c*<esc>},
\code{<esc>c+<esc>} and \code{<esc>dsC<esc>} respectively.} below:
\begin{mma}
\begin{inm}
DecomposeRepresentation[A2][Irrep[A2][\{1,0\}] $\tensor$
Irrep[A2][\{0,1\}]]
\end{inm}
\begin{outm}
$\Complex$ $\directSum$ Irrep[A2][\{1,1\}]
\end{outm}
\begin{inm}
WeightMultiplicities[F4,Irrep[F4][\{0,0,0,1\}]]
\end{inm}
\begin{outm}
???
\end{outm}
\end{mma}

\section{Explicit representations}
Perhaps the most important function in the \pkg{QuantumGroups} package is
\code{MatrixPresentation}, which produces explicit matrices representing
the action of the quantum group generators on a representation.

In order to understand how these are produced, we need to make use of the
following two results:

\begin{itemize}
\item Every irrep of $\uqg$ is a subrepresentation of some tensor product
of fundamental representations.
\item Every fundamental representation is subrepresentation of some
tensor product of `minuscule representations' and `short root
representations'.
\end{itemize}

The first result is trivial; to produce the irrep with highest weight
$\lambda = (\lambda_1, \lambda_2, \ldots, \lambda_n)$, pick high weight
vectors $v_i$ in each fundamental representation $V_{e_i}$, and look at
$\uqg^-\left(\Tensor_{i=1}^n  v_i^{\tensor \lambda_i}\right) \subset
\Tensor_{i=1}^n  V_{e_i}^{\tensor \lambda_i}$. This is an irrep,
generated by a high weight vector, and so must be what we want.

We'll explain now what `minuscule' and `short root' representations are,
and explain the easy proof of the second result. I was unable to find a
reference for this statement. Although it is unsurprising, it's essential
to what follows that every representation can be found inside tensor
products of representations which we can present as explicitly as we can
the minuscule and short root representations.

There are several equivalent characterisations of a minuscule
representation. The simplest to state is ... ???

What is a short root representation ... ???

The following representations are minuscule ...

The following representations are short root representations ...

This is how we find every other fundamental representation inside tensor
products of these ...

Here is what minuscule representations look like

Here is what short root representations look like, cf Jantzen.

It is invoked as \code{MatrixPresentation$[\Gamma][Z][V, \lambda,
\beta]$}. Here
\begin{defn}
\item[$\Gamma$] is the Cartan type, see \S \ref{sec:cartan-types}.
\item[$Z$] is a generator of the quantum group $\Gamma$, that is
$X_i^\pm$ or $K_i$, for $1 \leq i \leq \rank{\Gamma}$. Compositions of
generators, in the notation of \ref{sec:composition}, and linear
combinations, are also allowed. (Linear combinations must be homogeneous
with respect to the weight grading.)
\item[$V$] is a representation, in the notation of \S
\ref{sec:representations}.
\item[$\lambda$] is a weight, in the notation of \S \ref{sec:weights};
that is, a vector of integers, giving the weight as a linear combination
of fundamental weights.
\item[$\beta$] is a symbol specifying the basis to use. Possible options are described in \S \ref{sec:bases}, but nearly always you'll use
\code{FundamentalBasis}.
\end{defn}

\subsection{Bases}
The function \code{MatrixPresentation} takes an argument specifying the
desired basis. In the current implementation, there is only one useful
value -- the symbol \code{FundamentalBasis}. While we give a description
of how this basis is recursively defined below, essentially it depends on
many minor details of the implementation. One should not depend on any
particular properties of this basis!

Future versions of the \pkg{QuantumGroups} package may allow the use of
the symbols \code{GelfandTsetlinBasis} and \code{CanonicalBasis}, with
the obvious results.\footnote{Gelfand-Tsetlin bases are only projectively
canonical.} Code implementing Gelfand-Testlin bases exists, but is not
currently part of the package. Anyone interested in adding support for
canonical bases should certainly contact me!

\section{Invariant vectors and intertwiners}

\section{$R$-matrices and quantum knot invariants}
