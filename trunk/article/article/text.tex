In this paper, we describe a \MMA package for performing computations in
the representation theory of an arbitrary quantum group $\uqg$. Subject
to quite restrictive practical limitations,\footnote{My code is
inefficient, the algorithms are slow, and the computations are
difficult!} this package can
\begin{itemize}
\item Calculate dimensions of weight spaces and invariant spaces of
tensor products of arbitrary highest weight representations, using a
combinatorial model.
\item Produce matrices representing the action of the generators of the quantum group
$\uqg$ on an arbitrary highest weight representation.
\item Calculate bases for the invariants spaces inside tensor
products of representations, or bases for intertwining maps between
two such tensor products.
\item Calculate the action of the universal $R$-matrix on pairs of
representations.
\end{itemize}
As an application, we demonstrate the computation of
Reshetikhin-Turaev quantum knot invariants in many cases which have
not been previously calculated.

This paper is instructional, rather than expository; we give
examples of the usage of the package, but present no `new
mathematics'.

\section{Installing the \code{QuantumGroups`} package}

\section{Combinatorial representation theory}

\section{Explicit representations}
The function \code{MatrixPresentation} produces explicit matrices
representing the action of the quantum group generators on a
representation.

It is invoked as \code{MatrixPresentation[\Gamma][Z][V, \lambda, \beta]}.
Here
\begin{defn}
\item[$\Gamma$] is the Cartan type, see \S \ref{sec:cartan-types}.
\item[$Z$] is a generator of the quantum group $\Gamma$, that is
$X_i^\pm$ or $K_i$, for $1 \leq i \leq \rank{\Gamma}$. Compositions of
generators, in the notation of \ref{sec:composition}, and linear
combinations, are also allowed. (Linear combinations must be homogeneous
with respect to the weight grading.)
\item[$V$] is a representation, in the notation of \S
\ref{sec:representations}.
\item[$\lambda$] is a weight, in the notation of \S \ref{sec:weights};
that is, a vector of integers, giving the weight as a linear combination
of fundamental weights.
\end{defn}

\subsection{Bases}
The function \code{MatrixPresentation} takes an argument specifying the
desired basis. In the current implementation, there is only one useful
value -- the symbol \code{FundamentalBasis}. While we give a description
of how this basis is recursively defined below, essentially it depends on
many minor details of the implementation. One should not depend on any
particular properties of this basis!

Future versions of the \code{QuantumGroups`} package may allow the use of
the symbols \code{GelfandTsetlinBasis} and \code{CanonicalBasis}, with
the obvious results.\footnote{Gelfand-Tsetlin bases are only projectively
canonical.} Code implementing Gelfand-Testlin bases exists, but is not
currently part of the package. Anyone interested in adding support for
canonical bases should certainly contact me!

\section{Invariant vectors and intertwiners}

\section{$R$-matrices and quantum knot invariants}
